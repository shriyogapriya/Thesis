\documentclass[twoside,a4paper]{report}
\usepackage[T1]{fontenc}
\usepackage[utf8]{inputenc}
\usepackage{hyperref}
\usepackage{graphicx}
\usepackage{titling}
\graphicspath{ {./images/} }
\newcommand{\subtitle}[1]{%
	\posttitle{%
		\par\end{center}
	\begin{center}\LARGE#1\end{center}
	\vskip0.5em}%
}

\title{ALBERT LUDWIGS UNIVERISTY OF FREIBURG
		\\{\Large Master Thesis}}
\subtitle{\line(1,0){250}\\{\textbf{\Huge Multisite RNA-RNA Interaction Prediction}}\\\line(1,0){250}}	

\author{Yogapriya Ayyanarmoorthy}
\date{\today}

\begin{document}
	
	\maketitle
	
	\tableofcontents
	
	\chapter{Introduction}
	RNA molecules play important roles in various biological processes.Their regulation and function are mediated by interacting with other molecules. 	Forming base pairs between two RNAs, called RNA-RNA interactions (RRI). There are fast and reliable single interaction site (S-RRI) prediction tools like IntaRNA, that often show the additional sites within their suboptimal list, ie. are capable of modelling all sites individually but not in a joint prediction. Many RNAs interact via multiple synchronous, non-overlapping subinteractions (M-RRI), e.g. OxyS-fhlA. The simultaneous prediction of both intra- and inter-molecular base pairing allowing for multiple sites is computationally expensive. Some known approaches are IRIS, piRNA, NUPACK. Here we use a S-RRI prediction tool (namely IntaRNA) for the prediction of M-RRI.
	
	\section{Biological Background of RNA}
	\begin{itemize}
	 	\item 1.What is RNA
	 	\item 2.RNA representation {a,c,g,u}
	 	\item 3.classes of rna
	 	\item 4.base pairs of RNA
	 	\item 5.RNA secondary structure
	 	\item 6.types of rna secondary structure
	 	\item 7.nearest neighbor model
	 	\item 8.unpair probabilities 
	\end{itemize}
	
	\section{RNA-RNA Interaction}
 	Computational prediction of RNA-RNA interactions (RRI) is a central methodology for the specific investigation of inter-molecular RNA interactions and regulatory effects of non-coding RNAs.RNA–RNA interactions are fast emerging as a major functional component in many newly discovered non-coding RNAs.\\
 	
 	\begin{itemize}
 	\item Why RRI
 	\end{itemize}

	\section{RNA-RNA Interaction Prediction Approaches}
	There are several available methods, that can be classified according to their underlying prediction strategies, each implicating unique capabilities and restrictions often not transparent to the non-expert user.\\ 
	Most computational methods for RNA structure or RNA-RNA interaction prediction
	are based on thermodynamic models and provide an efficient computation since Richard Bellman’s principle of optimality \cite{raden2018interactive} can be applied.
	
	\begin{itemize}
	\item 1. Approaches that predict RRI
	\end{itemize}
	
	\subsection{Hybrid}
	\subsection{General}
	\subsection{Concatenation}
	\subsection{Accessibility}
		\begin{itemize}
		\item 1. S-RRI, M-RRI
		\item 2. problems with S-RRI
	    \end{itemize}
	\subsection{Adv. and disadv.}
		Hence we go for, Multi-site RRI optimization based on single-site IntaRNA predictions.
		
	\chapter{Multisite Accessibility Based  }
	\chapter{Results}
	\chapter{Discussion and conclusion}
	
	
	\bibliographystyle{plain}
	\bibliography{ref}

	
\end{document}